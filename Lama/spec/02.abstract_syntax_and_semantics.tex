\chapter{Abstract Syntax and Semantics}

In this section we present abstract syntax and semantics of \lama. The syntax is given in the form of \emph{abstract syntax tree} (AST) format,
and the semantics~--- in the form of a \emph{big-step operational semantics}. In addition we present a part of static semantics of the language
in the form of \emph{attribute system}, which can be considered as a form of simplictic type system. Only the core constructs of the languge
are considered here.

Basic syntactic and semantic categories and their brief descriptions are given in Tab.~\ref{categories}. Apart from those, we will use the
following notations: for a mapping $f : X\to Y$ we use the following definition:

\[
f [x\gets y] = \lambda\,z\,.\,
\left\{
\begin{array}{rcl}
  y    &,& x = z \\
  f\;x &,& x\neq z
\end{array}
\right.
\]

Empty mapping (undefined everywhere) is denoted $\Lambda$, the domain of a mapping $f$~--- $\primi{dom}{f}$, and we abbreviate

\[
  \Lambda[x_1\gets y_1][x_2\gets y_2]\dots[x_k\gets y_k]
\]

as

\[
  [x_1\gets y_1,\,x_2\gets y_2,\,\dots,\,x_k\gets y_k]
\]

For lists, we use notation $h\circ t$ for decomposition lists into heads and tails, $\epsilon$~--- for empty lists,
$|l|$~--- for list length and

\[
l[i:j]\qquad l[:n]\qquad l[n:]
\]

for taking sublists from $i$-th to $j$-th elements, $n$-last and $n$-first elements respectively; the numbering
of elements is left-to-right, starting from $0$. We also will enclose elements of lists which have a
composite structure in square brackets ``$[\dots]$'' to separate them properly.

\begin{table}[t]
  \begin{tabular}{cccl}
    notation           & instances                       & definition                                              & comments \\[1mm]
    \hline\\
    $\inbr{\bullet,\,\bullet,\,\dots}$ & & & pairs and tuples\\[1mm]
    $X^\bot$                            & & & $X\cup\{\bot\}$\\[1mm]
    $X^*$                              & & & lists of elements of $X$\\[1mm]    
    $\mathscr X$       & $x,\,y,\,z,\,\dots$             &                                                         & variables \\[1mm]
    $\mathscr E$       &                                 &                                                         & expressions \\[1mm]
    $\mathscr T$       & $\con{C},\,\con{D},\,\dots$     &                                                         & tags (constructors) \\[1mm]
    $\Sigma$           & $\sigma$                        & $\mathscr X\to\mathscr V$                               & bindings (a partial map from\\
                       &                                 &                                                         & variables to values) \\[1mm]
    $\Sigma_{\mathscr X}$ & $\inbr{S,\,\sigma}$             & $2^{\mathscr X}\times\Sigma$                              & local scopes (sets of variables\\
                       &                                 &                                                         & and their bindings) \\[1mm]
    $St$               & $\inbr{\sigma_g,\,ss}$          & $\Sigma\times\Sigma^*_{\mathscr X}$                        & states (global bindings and stacks\\        
                       &                                 &                                                         & of local scopes) \\[1mm]
    $\mathcal W$       & $w$                             &                                                         & worlds\\[1mm]
    $\mathcal L$       & $l$                             &                                                         & locations \\[1mm]
    $\mathcal M$       & $\mu$                           & $\mathcal L\to\mathcal C$                               & abstract memory (a partial map from\\
                       &                                 &                                                         & locations to composite values) \\[1mm]
    $\mathcal C$       & $c$                             & $St\times\mathcal{M}\times\mathcal{W}$                  & configurations (state, memory and world)\\[2mm]
    $\mathcal P$       &                                 & $\mathbb Z^\bot\uplus$                                   & primitive values (integer or default\\
                       &                                 & $\primi{ref}{\mathscr{X}}\uplus$                        & value, references to variables or\\
                       &                                 & $\primi{elemRef}{\mathcal L\;\mathbb N}$                & array elements)\\[2mm]
    $\mathcal F$       &                                 & $\mathscr X^*\times\mathscr E \times \mathbb N$         & function values (argument names, body,\\
                       &                                 &                                                         & nesting level)\\[1mm]
    $\mathcal V$       & $v$                             & $\mathcal P\uplus \mathcal L$                           & values (primitive values or locations) \\[1mm]
    $\mathcal C$       &                                 & $Arr\uplus Sexp \uplus Clo$                             & composite values (arrays, S-expressions, \\
                       &                                 &                                                         & or closures) \\[1mm]
    $Arr$              &                                 & $\mathbb N\times (\mathbb N\to\mathcal V)$              & arrays (length and element function) \\[1mm]
    $Sexp$             &                                 & $\mathscr T \times Arr$                                 & S-expressions (tag and array of subvalues) \\[1mm]
    $Clo$              &                                 & $\mathscr X^* \times\mathscr E\times\Sigma^*_{\mathscr X}$ & closures (argument names, function body\\
                       &                                 &                                                         & and a stack of local scopes) 
  \end{tabular}
  \caption{Basic Categories}
  \label{categories}
\end{table}

\FloatBarrier
\section{Abstract Syntax}

Abstract syntax defines the shapes of \emph{abstract syntax trees} (AST). We give the formation rules in a \emph{mixfix} form to
reflect the concrete syntax (presented in Section~\ref{sec:concrete_syntax}); however, the intended meaning is \emph{prefix},
thus the formation rule for, for example, conditional expression

\[
\mbox{\llang{if $\;\mathscr E\;$ then $\;\mathscr E\;$ else $\;\mathscr E$}}
\]

defines the following AST fragment:

\[
\mbox{\llang{if$\,(\mathscr E,\,\mathscr E,\,\mathscr E)$}}
\]

The root syntactic category for the language is \emph{compilation unit} $\mathscr U$, which is an expression in a contexts of
a group of top-level \emph{definitions} $\mathscr D$:

\[
\begin{array}{rcll}
  \mathscr S & :: & \mathscr D^*\quad \mathscr E & \mbox{--- scope expression}\\
  \mathscr D & :: & \mbox{\lstinline|var $\;\mathscr X$|} & \mbox{--- local variable definition} \\
             &    & \mbox{\lstinline|fun $\;\mathscr X\;$ ($\mathscr X^*$) \{$\;\mathscr S\;$\}|} & \mbox{--- function definition}\\
  \mathscr U & :: & \mathscr S & \mbox{--- compilation unit}
\end{array}
\]

The core syntactic category for the language is \emph{expressions} $\mathscr E$. The following kinds of
expressions are distinguished:

\[
\begin{array}{rcll}
   \mathscr E & :: & \mathscr X                                                            & \mbox{--- variable}  \\
              &    & \mathbb N                                                             & \mbox{--- constant}  \\
              &    & \lstinline|{ $\mathscr S\;$ }|                                        & \mbox{--- nested scope} \\   
              &    & \mathscr E \otimes \mathscr E                                         & \mbox{--- binary expression} \\
              &    & \mathscr E\;\lstinline|($\mathscr E^*$)|                              & \mbox{--- call} \\
              &    & \llang{fun ($\mathscr X^*$) \{$\mathscr S$\}}                         & \mbox{--- lambda expression} \\
              &    & \llang{[$\mathscr E^*$]}                                              & \mbox{--- array} \\
              &    & \llang{$\mathscr E\;$ [$\mathscr E$]}                                 & \mbox{--- array element} \\
              &    & \llang{$\mathscr T\;$ ($\mathscr E^*$)}                               & \mbox{--- S-expression}  \\
              &    & \mathscr E \;\mbox{\lstinline|:=|} \;\mathscr E                       & \mbox{--- assignment} \\
              &    & \mathscr E \mbox{\lstinline|;|}\; \mathscr E                          & \mbox{--- sequential expression} \\
              &    & \llang{if $\;\mathscr E\;$ then $\;\mathscr E\;$ else $\;\mathscr E$} & \mbox{--- conditional expression} \\
              &    & \llang{while $\;\mathscr E\;$ do $\;\mathscr E$}                      & \mbox{--- loop expression} \\
              &    & \llang{case $\;\mathscr E\;$ ($\mathscr P$ -> $\mathscr E$)$^+$}      & \mbox{--- pattern matching} \\
              &    & \mbox{\lstinline|skip|}                                               & \mbox{--- empty expression} \\
              &    & \bot                                                                  & \mbox{--- default value} \\
              &    & \llang{ref ($\mathscr X$)}                                            & \mbox{--- reference to variable} \\
              &    & \llang{elemRef ($\mathscr E$,$\;\mathscr E$)}                         & \mbox{--- reference to array element}  \\
              &    & \llang{ignore ($\mathscr E$)}                                         & \mbox{--- ignore expression}
\end{array}
\]

Note, in the formation rule for binary expressions the symbol ``$\otimes$'' is left undefined. The actual assortment of
binary operators is given in Section~\ref{sec:expressions}.

The last four kinds of expressions (``$\bot$'', \llang{ref}, \llang{elemRef} and \llang{ignore}) are \emph{internal} in the sense that
they are not represented  explicitly in the concrete syntax; instead, they are \emph{inferred} using the attribute system described
in Section~\ref{sec:wellformedness}. 

The last category is \emph{patterns} $\mathscr P$. Patterns are used in \lstinline|case|-expressions to match the values against:

\[
\begin{array}{rcll}
  \mathscr P & :: & \_                                     & \mbox{--- wildcard} \\
             &    & \mathbb N                              & \mbox{--- constant} \\
             &    & \llang{[$\mathcal P^*$]}               & \mbox{--- array}\\
             &    & \llang{$\mathcal T\;$($\mathcal P^*$)} & \mbox{--- S-expression}\\
             &    & \llang{array}                          & \mbox{--- array type}\\
             &    & \llang{sexp}                           & \mbox{--- S-expression type}\\
             &    & \llang{fun}                            & \mbox{--- closure type}\\  
             &    & \llang{val}                            & \mbox{--- value type}\\             
             &    & \llang{box}                            & \mbox{--- reference type}
\end{array}
\]


\section{Operational Semantics}

The semantics for the language is given in the form of a \emph{big-step} operational semantics with the
main transition relation being

\[
\setarrow{\xRightarrow}
\trans{c}{e}{\inbr{c^\prime,\,v}}
\]

for some \emph{configurations} $c, c^\prime$, some expression $e$ and some value $v$. Besides this main relation a few supplementary are
used to handle the semantics of other syntactic categories.

\subsection{States, Worlds and Configurations}

A configuration is a triple of a \emph{state}, a \emph{memory}, and a \emph{world}. A world is an abstract entity to accomodate side-effects
potentially performed as the execution of a program proceeds, and a memory is a patrial function which maps \emph{locations} to
\emph{composite values} (arrays, S-expressions, and closures).

A state is a pair

\[
\inbr{\sigma_g,\,ss}
\]

of a \emph{global} binding and a stack (list) of local scopes, each of which is a pair

\[
\inbr{S,\,\sigma}
\]

where $S$ is a set of variable names and $\sigma$ is local bindings. For states we define two primitives to read a variable
value in a state 


\[
\begin{array}{rcl}
  \inbr{\sigma_g,\,\epsilon}\,x & = &\sigma_g\,x\\[2mm]
  \inbr{\sigma_g,\,\inbr{S,\,\sigma}\circ ss}\,x  & = & \left\{\begin{array}{rcl}
                                                                  \sigma\,x &,& x\in S\\
                                                                  \inbr{\sigma_g,\,ss}\,x &,&x\not\in S
                                                               \end{array}
                                                        \right.
\end{array}
\]

and to reassign a variable to another value:

\[
\begin{array}{rcl}
  \inbr{\sigma_g,\,\epsilon}\,[x\gets v] & = &\inbr{\sigma_g\,[x\gets v],\,\epsilon}\\[2mm]
  \inbr{\sigma_g,\,\inbr{S,\,\sigma}\circ ss}\,[x\gets v]  & = & \left\{\begin{array}{rcl}
                                                                          \inbr{\sigma_g,\,\inbr{S,\,\sigma\,[x\gets v]}\circ ss} &,& x\in S\\[1mm]
                                                                          \inbr{\sigma^\prime_g,\,\inbr{S,\,\sigma}\circ ss^\prime}\,x &,&x\not\in S\\
                                                                                                                                    & &\inbr{\sigma^\prime_g,\,ss^\prime}=\inbr{\sigma_g,\,ss}\,[x\gets v]
                                                                        \end{array}
                                                                 \right.
\end{array}
\]

\subsection{Expressions}

To give the semantics for the expressions we introduce a supplementary transition ``$\xRightarrow{}^*$'', which described a left-to-right computation
of the \emph{list} of expressions and results, correspondingly, in a final configuration and a \emph{list} of values:

\setarrow{\xRightarrow}
\setsubarrow{\hskip-0,3em^*\hskip0.3em}
\[ 
\trans{c}{\epsilon}{\inbr{c,\,\epsilon}}\ruleno{Expr$^*_\epsilon$} 
\]
\[ 
\trule{{\setsubarrow{}\trans{c}{e}{\inbr{c^\prime,\,v}}}\qquad\trans{c^\prime}{\omega}{\inbr{c^{\prime\prime},\,\psi}}}
      {\trans{c}{e\circ\omega}{\inbr{c^{\prime\prime},\,v\circ\psi}}}\ruleno{Expr$^*$} 
\]

Operational semantics for core expressions is given in Fig.~\ref{semantics:core}. Note, the semantics of binary expressions is
given in terms of a primitive ``$\oplus$'', which is left undefined. The concrete semantics of binary operators is
given in Section~\ref{sec:expressions}.

\begin{figure}[t]
\setsubarrow{}
\[
\trans{c}{z}{\inbr{c,\,z}}\ruleno{Const}
\]
\[
\trans{\inbr{s,\,m,\,w}}{x}{\inbr{\inbr{s,\,m,\,w},\,s\,x}}\ruleno{Var}
\]
\[
\trans{c}{\llang{ref $\;x$}}{\inbr{c,\,\primi{ref}{x}}}\ruleno{Ref}
\]
\[
\trule{\setsubarrow{\hskip-0,3em^*\hskip0.3em}\trans{c}{l\circ r}{\inbr{c^\prime,\,w\circ v}}}
      {\trans{c}{l\oplus r}{\inbr{c^\prime,\,w\otimes v}}}\ruleno{Binop}
\]\vskip1mm
\[
\trans{c}{\llang{skip}}{\inbr{c,\,\bot}}\ruleno{Skip}
\]
\[
\trule{\setsubarrow{\hskip-0,3em^*\hskip0.3em}\trans{c}{l\circ r}{\inbr{\inbr{s,\,m,\,w},\,[\primi{ref}{x}]\circ[v]}}}
      {\trans{c}{\llang{$l\;$ := $\;r$}}{\inbr{\inbr{s\,[x\gets v],\,m,\,w},\,v}}}\ruleno{Assign}
\]\vskip1mm
\[
\trule{\trans{c_1}{S_1}{\inbr{c^\prime,\,v}}\qquad \trans{c^\prime}{S_2}{c_2}}
      {\trans{c_1}{S_1\llang{;}\;S_2}{c_2}}\ruleno{Seq}
\]\vskip1mm
\[
\trule{\trans{c}{e}{\inbr{c^\prime,\,n}}\qquad n\ne 0\qquad \trans{c^\prime}{S_1}{c^{\prime\prime}}}
      {\trans{c}{\llang{if $\ \ e\ \ $ then $\ \ S_1\ \ $ else $\ \ S_2\ \ $}}{c^{\prime\prime}}}
      \ruleno{If-True}
\]\vskip1mm
\[
\trule{\trans{c}{e}{\inbr{c^\prime,\,0}}\qquad \trans{c^\prime}{S_2}{c^{\prime\prime}}}
      {\trans{c}{\llang{if $\ \ e\ \ $ then $\ \ S_1\ \ $ else $\ \ S_2\ \ $}}{c^{\prime\prime}}}
      \ruleno{If-False}
\]\vskip1mm
\[
\trule{\trans{c}{e}{\inbr{c^\prime,\,n}}\quad n\ne 0\quad\trans{c^\prime}{S}{\inbr{c^{\prime\prime},\,v}}\quad\trans{c^{\prime\prime}}{\llang{while $\ \ e\ \ $ do $\ \ S\ \ $}}{c^{\prime\prime\prime}}}     
      {\trans{c}{\llang{while $\ \ e\ \ $ do $\ \ S\ \ $}}{c^{\prime\prime\prime}}}
      \ruleno{While-True}
\]\vskip1mm
\[
\trule{\trans{c}{e}{\inbr{c^\prime,\,0}}}
      {\trans{c}{\llang{while $\ \ e\ \ $ do $\ \ S\ \ $}}{\inbr{c^\prime,\,\bot}}}
      \ruleno{While-False}
\]
\caption{Big-step Operational Semantics for Core Expressions}
\label{semantics:core}
\end{figure}

\subsubsection{Arrays and S-expressions}

Arrays and S-expressions allow to manipulate \emph{composite valuues} which are kept in \emph{memory}. A memory is a partial function $m$
which maps \emph{locations} into composite values. Locations are abstractions of real-world memory addresses; the set of locations
is considered to be linearly ordered. We define the following primitive

\[
\primi{new}\,m\,v=\inbr{l,\,m\,[l\gets v]}\mbox{ where }l=min\{l\in\mathcal{L}\mid m\,(l)\mbox{ undefined}\}
\]

which takes a memory $m$ and a composite value $v$ and returns a \emph{new} location $l$ (previously unoccupied) and a new \emph{memory} which
associates $l$ with $v$ and preserves other associations intact.

Arrays are represented as pairs

\[
\inbr{n,\,f}
\]

where $n\in\mathbb N$ is a lengths and $f : \mathbb{N}\to\mathcal{V}$~--- element function which maps indices into values. It is assumed, that
$\primi{dom}{f}=[0\dots n-1]$.

S-expressions are represented as pairs

\[
\inbr{t,\,s}
\]

where $t$ is a \emph{tag} and $s$ is an array of subexpressions. The semantics of arrays and subexpressions is given in Fig.~\ref{semantics:arrays}.

\begin{figure}
  \setsubarrow{}
  \[
  \trule{\setsubarrow{\hskip-0,3em^*\hskip0.3em}
         \begin{array}{c}
           \trans{c}{e_1\circ\dots\circ e_k}{\inbr{\inbr{s,\,m,\,w}, v_1\circ\dots\circ v_k}}\\[2mm]
           \inbr{l,\,m^\prime}=\primi{new}{m\,\inbr{k-1,\,i\mapsto v_{i+1}}}
         \end{array}
        }
        {\trans{c}{\llang{[$e_1,\dots,e_k$]}}{\inbr{\inbr{s,\,m^\prime,\,w},\,l}}}\ruleno{Array}
  \]\\[2mm]
  \[
  \trule{\setsubarrow{\hskip-0,3em^*\hskip0.3em}
         \begin{array}{c}
           \trans{c}{e_1\circ\dots\circ e_k}{\inbr{\inbr{s,\,m,\,w}, v_1\circ\dots\circ v_k}}\\[2mm]
           \inbr{l,\,m^\prime}=\primi{new}{m\,\inbr{t,\,\inbr{k-1,\,i\mapsto v_{i+1}}}}
         \end{array}
        }
        {\trans{c}{\llang{$t\,$($e_1,\dots,e_k$)}}{\inbr{\inbr{s,\,m^\prime,\,w},\,l}}}\ruleno{Sexp}  
  \]\\[2mm]
  \[
  \trule{\setsubarrow{\hskip-0,3em^*\hskip0.3em}
         \begin{array}{c}
           \trans{c}{e_1\circ e_2}{\inbr{\inbr{s,\,m,\,w},\,l\circ i}}\\[2mm]
           l\in\mathcal{L}\\
           m\,(l)=\inbr{k,\,f}\mbox{ or }m\,(l)=\inbr{t,\,\inbr{k, f}}\\
           i\in[0\dots k-1]
         \end{array}
        }
        {\trans{c}{\llang{$e_1\,$[$e_2$]}}{\inbr{\inbr{s,\,m,\,w},\,f\,(i)}}}\ruleno{Elem}
  \]\\[2mm]
  \[
  \trule{\setsubarrow{\hskip-0,3em^*\hskip0.3em}
         \begin{array}{c}
           \trans{c}{e_1\circ e_2}{\inbr{\inbr{s,\,m,\,w},\,l\circ i}}\\[2mm]
           l\in\mathcal{L}\\
           m\,(l)=\inbr{k,\,f}\mbox{ or }m\,(l)=\inbr{t,\,\inbr{k, f}}\\
           i\in[0\dots k-1]
         \end{array}
        }
        {\trans{c}{\llang{elemRef ($e_1$,$\ \ e_2$)}}{\inbr{\inbr{s,\,m,\,w},\,\primi{elemRef}{l\;i}}}}\ruleno{ElemRef}
  \]\\[2mm]
  \[
  \trule{\setsubarrow{\hskip-0,3em^*\hskip0.3em}
         \begin{array}{c}
           \trans{c}{l\circ r}{\inbr{\inbr{s,\,m,\,w},\,[\primi{elemRef}{l\;i}]\circ[v]}}\\[2mm]
           m^\prime=\left\{\begin{array}{rcl}
                            m\,[l\gets\inbr{k,\,f\,[i\gets v]}] &,&m\,(l)=\inbr{k,\,f}\\
                            m\,[l\gets\inbr{t,\,\inbr{k,\,f\,[i\gets v]}}] &,&m\,(l)=\inbr{t,\,\inbr{k,\,f}}
                          \end{array}
                   \right. 
         \end{array}
        }
        {\trans{c}{\llang{$l\;$ := $\;r$}}{\inbr{\inbr{s,\,m,\,w},\,v}}}\ruleno{AssignElem}
  \]    
\caption{Big-step Operational Semantics for Arrays and S-expressions}  
\label{semantics:arrays}
\end{figure}

\subsubsection{Pattern Matching}

\subsubsection{Declarations}

\subsubsection{Functions and Closures}


%\section{Well-formed Expressions}
\label{sec:wellformedness}

{
\renewcommand{\Ref}{\primi{Ref}{}}
\newcommand{\Val}{\primi{Val}{}}
\newcommand{\Void}{\primi{Void}{}}
\newcommand{\Weak}{\primi{Weak}{}}
\renewcommand{\withenv}[2]{{#2}\,:\,{#1}}

\begin{comment}
    \withenv{\Ref}{\llang{ref $\;x$}} & \withenv{\Val}{x}& \withenv{\Void}{\llang{ignore $\;x$}} & x \in \mathscr X\\
                                      & \withenv{\Val}{z}& \withenv{\Void}{\llang{ignore $\;z$}} & z \in \mathbb N \\
                                      & \trule{\withenv{\Val}{l},\quad\withenv{\Val}{r}} 
                                              {\withenv{\Val}{l\oplus r}} &
                                        \trule{\withenv{\Val}{l},\quad\withenv{\Val}{r}}
                                              {\withenv{\Void}{\llang{ignore $\;l\oplus r$}}}    & \\
                                      &                  & \withenv{\Void}{\llang{skip}}         & \\
                                      & \trule{\withenv{\Ref}{l},\quad\withenv{\Val}{r}}
                                              {\withenv{\Val}{\llang{$l\;$ := $\; r$}}} &
                                        \trule{\withenv{\Ref}{l},\quad\withenv{\Val}{r}}
                                              {\withenv{\Void}{\llang{ignore ($l\;$ := $\; r$})}} & \\
                                      &                  & \withenv{\Void}{\llang{read ($x$)}} & \\
                                      &                  & \trule{\withenv{\Val}{e}}{\withenv{\Void}{\llang{write ($e$)}}} & \\[2mm]
      \trule{\withenv{\Void}{s_1},\quad\withenv{a}{s_2}}{\withenv{a}{\llang{$s_1\;$;$\;s_2$}}}&
      \trule{\withenv{\Val}{e},\quad\withenv{a}{s_1},\quad\withenv{a}{s_2}}{\withenv{a}{\llang{if $\;e\;$ then $\;s_1\;$ else $\;s_2$}}}&
      \trule{\withenv{\Val}{e},\quad\withenv{\Void}{s}}{\withenv{\Void}{\llang{while $\;e\;$ do $\;s$}}}&       \\[2mm]
       & \trule{\withenv{\Val}{e},\quad\withenv{\Void}{s}}{\withenv{\Void}{\llang{repeat $\;s\;$ until $\;e$}}} & &
\end{comment}

\renewcommand{\arraystretch}{3}
\[
  \begin{array}{cc}
    \trule{x \in \mathscr X}{\withenv{\Weak}{x}}  & \trule{z \in \mathbb N }{\withenv{\Weak}{z}}\\
    
  \trule{\withenv{\Val}{l},\quad\withenv{\Val}{r}} 
        {\withenv{\Weak}{l\oplus r}}               & \withenv{\Weak}{\llang{skip; $\;\bot$}} \\
  \trule{\withenv{\Ref}{l},\quad\withenv{\Val}{r}}
        {\withenv{\Weak}{\llang{$l\;$ := $\; r$}}} & \withenv{\Weak}{\llang{read ($x$); $\;\bot$}} \\
  \trule{\withenv{\Val}{e}}{\withenv{\Weak}{\llang{write ($e$); $\;\bot$}}} & \trule{\withenv{\Val}{e},\quad\withenv{\Void}{s}}{\withenv{\Weak}{\llang{while $\;e\;$ do $\;s\;$ od; $\;\bot$}}} \\
  \trule{\withenv{\Val}{e},\quad\withenv{\Void}{s}}{\withenv{\Weak}{\llang{(repeat $\;s\;$ until $\;e$); $\;\bot$ }}} & 
  \end{array}
  \]

}


