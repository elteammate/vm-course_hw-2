\chapter{Concrete Syntax and Semantics}
\label{sec:concrete_syntax}

In this chapter we describe the concrete syntax of the language as it is recognized by the parser. In the
syntactic description we will use the extended Backus-Naur form with the following conventions:

\begin{itemize}
\item nonterminals are presented in \nonterm{italics};
\item concrete terminals are \term{grayed out};
\item classes of terminals are \token{CAPITALIZED};
\item a postfix ``$^\star$'' designates zero-or-more repetitions;
\item square brackets ``$[\dots]$'' designate zero-or-one repetition;
\item round brackets ``$(\dots)$'' are used for grouping;
\item alteration is denoted by ``$\alt$'', sequencing by juxtaposition;
\item a colon ``$:$'' separates a nonterminal being defined from its definition.
\end{itemize}

In the description below we will take in-line code samples in blockquotes "..." which are not considered as a
part of the concrete syntax.

% !TEX TS-program = pdflatex
% !TeX spellcheck = en_US
% !TEX root = lama-spec.tex

\section{Lexical Structure}
\label{sec:lexical_structure}

The character set for the language is \textsc{ASCII}, case-sensitive. In the following lexical description we will use
the POSIX-Extended Regular Expressions in lexical definitions.

\subsection{Whitespaces and Comments}

Whitespaces and comments are \textsc{ASCII} sequences which serve as delimiters for other tokens but otherwise are
ignored.

The following characters are treated as whitespaces:

\begin{itemize}
\item blank character "\texttt{ }";
\item newline character "\texttt{\textbackslash n}";
\item carriage return character "\texttt{\textbackslash r}";
\item tabulation character "\texttt{\textbackslash t}".
\end{itemize}

Additionally, two kinds of comments are recognized:

\begin{itemize}
\item the end-of-line comment "\texttt{--}" escapes the rest of the line, including itself;
\item the block comment "\texttt{(*} ... \texttt{*)}" escapes all the text between
  "\texttt{(*}" and "\texttt{*)}".
\end{itemize}

There is a number of specific cases which have to be considered explicitly.

First, block comments can be properly nested. Then, the occurrences of comment symbols inside string literals (see below) are not
considered as comments.

End-of-line comment encountered \emph{outside} of a block comment escapes block comment symbols:

\begin{lstlisting}
    -- the following symbols are not considered as a block comment: (*
    -- same here: *)
\end{lstlisting}

Similarly, an end-of-line comment encountered inside a block comment is escaped:

\begin{lstlisting}
    (* Block comment starts here ...
       -- and ends here: *)
\end{lstlisting}

\subsection{Identifiers and Constants}

The language distinguishes identifiers, signed decimal literals, string and character literals (see Fig.~\ref{idents_and_consts}). There are
two kinds of identifiers: those beginning with uppercase characters (\token{UIDENT}) and lowercase characters (\token{LIDENT}).

String literals cannot span multiple lines; a blockquote character (") inside a string literal has to be doubled to prevent from
being considered as this literal's delimiter.

Character literals as a rule are comprised of a single \textsc{ASCII} character; if this character is a quote (') it has to be doubled. Additionally
two-character abbreviations "\textbackslash t" and "\textbackslash n" are recognized and converted into a single-character representation.

\begin{figure}[t]
  \[
  \begin{array}{rcl}
    \token{UIDENT} & = &\mbox{\texttt{[A-Z][a-zA-Z\_0-9]*}}\\
    \token{LIDENT} & = &\mbox{\texttt{[a-z][a-zA-Z\_0-9]*}}\\
    \token{DECIMAL}& = &\mbox{\texttt{-?[0-9]+}}\\
    \token{STRING} & = &\mbox{\texttt{"([\^{}\textbackslash"]|"")*"}}\\
    \token{CHAR}   & = &\mbox{\texttt{'([\^{}']|''|\textbackslash n|\textbackslash t)'}}
  \end{array}
  \]
  \caption{Identifiers and constants}
  \label{idents_and_consts}
\end{figure}


\subsection{Keywords}

The following identifiers are reserved for keywords:

\begin{lstlisting}
    after    array    at      before   box   case     do     elif     else
    esac     eta      false   fi       for   fun      if     import   infix
    infixl   infixr   lazy    od       of    public   sexp   skip     str
    syntax   then     true    val      var   while    let    in
\end{lstlisting}

\subsection{Infix Operators}

Infix operators defined as follows:

\[
\token{INFIX}=\mbox{\texttt{[+*/\%\$\#@!|\&\^{}~?<>:=\textbackslash-]+}}
\]

There is a predefined set of built-in infix operators (see Fig.~\ref{builtin_infixes}); additionally
an end-user can define custom infix operators (see Section~\ref{sec:custom_infix}). Note, sometimes 
additional whitespaces are required to disambiguate infix operator applications. For example, if a
custom infix operator "\lstinline|+-|" is defined, then the expression "\lstinline|a +- b|" can no longer be
recognized as "\lstinline|a +(-b)|". Note also that a custom operator containing "\lstinline|--|" can not be
defined due to lexical conventions.

\subsection{Delimiters}

The following symbols are treated as delimiters:

\begin{lstlisting}
    .       ,         (        )        {        }
    ;       #         ->       |
\end{lstlisting}

Note, custom infix operators can coincide with delimiters "\lstinline|#|", "\lstinline!|!", and "\lstinline|->|", which can
sometimes be misleading. 



% !TEX TS-program = pdflatex
% !TeX spellcheck = en_US
% !TEX root = lama-spec.tex
\FloatBarrier

\begin{figure}[t]
  \[
    \begin{array}{rcl}
      \defterm{compilationUnit}  & : & \nonterm{import}^\star\s\nonterm{scopeExpression}\\
      \defterm{import}           & : & \term{import}\s\token{UIDENT}\s\term{;}
    \end{array}
  \]
  \caption{Compilation unit concrete syntax}
  \label{compilation_unit}
\end{figure}

\section{Compilation Units}
\label{sec:compilation_units}

Compilation unit is a minimal structure recognized by the parser. An application can contain multiple units, compiled separately.
In order to use other units they have to be imported. In particular, the standard library is comprised of a number of precompiled units,
which can be imported by an end-user application.

The concrete syntax for compilation unit is shown in Fig.~\ref{compilation_unit}. Besides optional imports a unit must contain
a \nonterm{scopeExpression}, which may contain some definitions and computations. Note, a unit can not be empty. The computations described in
a unit are performed at the unit initialization time (see Chapter~\ref{sec:driver}).


% !TEX TS-program = pdflatex
% !TeX spellcheck = en_US
% !TEX root = lama-spec.tex
\begin{figure}[t]
  \[
    \begin{array}{rclc}
      \defterm{scopeExpression}                & : & \nonterm{definition}^\star\s[\s\nonterm{expression}\s]&\\
      \defterm{definition}                     & : & \nonterm{variableDefinition}&\alt\\
                                               &   & \nonterm{functionDefinition}&\alt\\
                                               &   & \nonterm{infixDefinition}&\\
      \defterm{variableDefinition}             & : & (\s\term{var}\alt\term{public}\s)\s\nonterm{variableDefinitionSequence}\s\term{;}&\\
      \defterm{variableDefinitionSequence}     & : & \nonterm{variableDefinitionItem}\s(\s\term{,}\s\nonterm{variableDefinitionItem}\s)^\star&\\
      \defterm{variableDefinitionItem}         & : & \token{LIDENT}\s[\s\term{=}\s\nonterm{basicExpression}\s]&\\
      \defterm{functionDefinition}             & : & [\s\term{public}\s]\s\term{fun}\s\token{LIDENT}\s\term{(}\s\nonterm{functionArguments}\s\term{)}&\\
                                               &   & \phantom{XXXXX}\nonterm{functionBody}&\\
      \defterm{functionArguments}              & : & [\s\token{LIDENT}\s(\s\term{,}\s\token{LIDENT}\s)^\star\s]&\\
      \defterm{functionBody}                   & : & \term{\{}\s\nonterm{scopeExpression}\s\term{\}}&
    \end{array}
  \]
  \caption{Scope expression concrete syntax}
  \label{scope_expression}
\end{figure}

\section{Scope Expressions}
\label{sec:scope_expressions}

Scope expressions provide a mean to put expressions is a scoped context. The definitions in scoped expressions comprise of function definitions and
variable definitions (see Fig.~\ref{scope_expression}). For example:

\begin{lstlisting}
    var x, y, z; -- variable definitions

    fun id (x) {x} -- function definition
\end{lstlisting}

As scope expressions are expressions, they can be nested:

\begin{lstlisting}
    var x;

    ( -- nested scope begins here
      var y;
      skip
    ) -- nested scope ends here
\end{lstlisting}

The definitions on the top-level of compilation unit can be tagged as ``\lstinline|public|'', in which case they are exported and become visible by
other units which import a given one. Nested scopes can not contain public definitions.

The nesting relation has the shape of a tree, and in a concrete node of the tree all definitions in all enclosing scopes are visible:

\begin{lstlisting}
    var x;

    (var y; 
      (var z;
        skip -- x, y, and z are visible here
      );
      (var t;
        skip -- x, y, and t are visible here
      );
      skip -- x and y are visible here
    );
    skip -- only x is visible here
\end{lstlisting}

Multiple definitions of the same name in the same scope are prohibited:

\begin{lstlisting}
    var x;
    fun x () {0} -- error
\end{lstlisting}

However, a definition is a nested scope can override a definition in an enclosing one:

\begin{lstlisting}
    var x;

    (
      fun x () {0} -- ok
      skip         -- here x is associated with the function
    );

    skip -- here x is associated with the variable
\end{lstlisting}

A function can freely use all visible definitions; in particular, functions defined in the
same scope can be mutually recursive:

\begin{lstlisting}
    var x;
    fun f () {0}

    ( 
      fun g () {f () + h () + y} -- ok
      fun h () {g () + x}        -- ok
      var y;
      skip
    );
    skip
\end{lstlisting}

A variable, defined in a scope, can be attributed with an expression, calculating its initial value.
These expressions, however, are evaluated in the order of variable declaration. Thus, while
technically it is possible to have forward references in the initialization expression, their
behavior is undefined. For example:

\begin{lstlisting}
    var x = y + 2; -- undefined, as y is not yet initialized at this point
    var y = x + 2;
    skip
\end{lstlisting}

% !TEX TS-program = pdflatex
% !TeX spellcheck = en_US
% !TEX root = lama-spec.tex
\section{Expressions}
\label{sec:expressions}

The syntax definition for expressions is shown in Fig.~\ref{expressions}. The top-level construct is \emph{sequential composition}, expressed
using right-associative connective "\term{;}". The basic blocks of sequential composition have the form of \nonterm{binaryExpression}, which is
a composition of infix operators and operands. The description above is given in a highly ambiguous form as it does not specify explicitly the
precedence and associativity of infix operators. The precedences and associativity of predefined built-in infix operators are shown
in Fig.~\ref{builtin_infixes} with the precedence level increasing top-to-bottom.

\begin{figure}[h]
  \begin{tabular}{c|l|l}
    infix operator(s) & description & associativity \\
    \hline
    \lstinline|:=|                                                                                & assignment                         & right-associative \\
    \lstinline|:|                                                                                 & list constructor                   & right-associative \\
    \lstinline|!!|                                                                                & disjunction                        & left-associative  \\
    \lstinline|&&|                                                                                & conjunction                        & left-associative  \\
    \lstinline|==|, \lstinline|!=|,  \lstinline|<=|, \lstinline|<|, \lstinline|>=|, \lstinline|>| & integer comparisons                & non-associative   \\
    \lstinline|+|, \lstinline|-|                                                                  & addition, subtraction              & left-associative  \\
    \lstinline|*|, \lstinline|/|, \lstinline|%|                                                   & multiplication, quotient, remainder & left-associative
  \end{tabular}
\caption{The precedence and associativity of built-in infix operators}
\label{builtin_infixes}
\end{figure}

Apart from assignment and list constructor all other built-in infix operators operate on signed integers; in conjunction and disjunction
any non-zero value is treated as truth and zero as falsity, and the result respects this convention.

The assignment operator is unique among all others in the sense that it requires its left operand to designate a \emph{reference}. This
property is syntactically ensured using an inference system shown in Fig.~\ref{reference_inference}; here $\mathcal{R}\,(e)$ designates the
property ``$e$ is a reference''. The result of assignment operator coincides with its right operand, thus

\begin{lstlisting}
    x := y := 3
\end{lstlisting}

assigns 3 to both "\lstinline|x|" and "\lstinline|y|".

\begin{figure}[h]
  \renewcommand{\Ref}[1]{\mathcal{R}\,({#1})}
  \renewcommand{\arraystretch}{4}
  \[
    \begin{array}{cc}
      \Ref{x},\,x\;\mbox{is a variable}&\dfrac{\Ref{e}}{\Ref{\lstinline|$e$ [$\dots$]|}}\\
      \dfrac{\Ref{e_i}}{\Ref{\mbox{\lstinline|if $\dots$ then $\;e_1\;$ else $\;e_2\;$ fi|}}} & \dfrac{\Ref{e_i}}{\Ref{\mbox{\lstinline|case $\dots$ of $\;\dots\;$ -> $\;e_1\;\dots\;\dots\;$ -> $\;e_k\;$ esac|}}}\\
      \multicolumn{2}{c}{\dfrac{\Ref{e}}{\Ref{\lstinline|$\dots\;$;$\;e$|}}}
    \end{array}
  \]
  \caption{Reference inference system}
  \label{reference_inference}
\end{figure}

\subsection{Postfix Expressions}

There are two postfix forms of expressions:

\begin{itemize}
\item function call, designated as postfix form "\lstinline|($arg_1, \dots, arg_k$)|";
\item array element selection, designated as "\lstinline|[$index$]|".
\end{itemize}

Also, see postfix ``dot'' notation (Section~\ref{sec:dot-notation}).

Multiple postfixes are allowed, for example

\begin{lstlisting}
    x () [3] (1, 2, 3) . string
    x . string [4]
    x . length . string
    x . string . length
\end{lstlisting}

The basic form of expression is \nonterm{primary}. The simplest form of primary is an identifier or constant. Keywords \lstinline|true| and \lstinline|false|
designate integer constants 1 and 0 respectively, character constant is implicitly converted into its \textsc{ASCII} code.  String constants designate arrays
of one-byte characters. Infix constants allow to reference a functional value associated with corresponding infix operator (however, a value associated with
builtin assignment operator "\lstinline|:=|" can not be taken), and functional constant (\emph{lambda-expression})
designates an anonymous functional value in the form of a closure. 

\begin{figure}[h]
  \[
    \begin{array}{rcll}
      \defterm{expression}        & : & \nonterm{basicExpression}\s(\s\term{;}\s\nonterm{expression}\s)&\\
      \defterm{basicExpression}   & : & \nonterm{binaryExpression}&\\
      \defterm{binaryExpression}  & : & \nonterm{binaryOperand}\s\token{INFIX}\s\nonterm{binaryOperand}&\alt\\
                                  &   & \nonterm{binaryOperand}&\\
      \defterm{binaryOperand}     & : & \nonterm{binaryExpression}&\alt\\
                                  &   & [\s\term{-}\s]\s\nonterm{postfixExpression}&\\
      \defterm{postfixExpression} & : & \nonterm{primary}&\alt\\
                                  &   & \nonterm{postfixExpression}\s\term{(}\s[\s\nonterm{expression}\s(\s\term{,}\s\nonterm{expression}\s)^\star\s]\s\term{)}&\alt\\
                                  &   & \nonterm{postfixExpression}\s\term{[}\s\nonterm{expression}\s\term{]}&\alt\\
      \defterm{primary}           & : & \token{DECIMAL}&\alt\\
                                  &   & \token{STRING}&\alt\\
                                  &   & \token{CHAR}&\alt\\
                                  &   & \token{LIDENT}&\alt\\
                                  &   & \term{true}&\alt\\
                                  &   & \term{false}&\alt\\
                                  &   & \term{infix}\s\token{INFIX}&\alt\\
                                  &   & \term{fun}\s\term{(}\s\nonterm{functionArguments}\s\term{)}\s\nonterm{functionBody}&\alt\\
                                  &   & \term{skip}&\alt\\
                                  &   & \term{(}\s\nonterm{scopeExpression}\s\term{)}&\alt\\
                                  &   & \nonterm{listExpression}&\alt\\
                                  &   & \nonterm{arrayExpression}&\alt\\
                                  &   & \nonterm{S-expression}&\alt\\
                                  &   & \nonterm{ifExpression}&\alt\\
                                  &   & \nonterm{whileDoExpression}&\alt\\
                                  &   & \nonterm{doWhileExpression}&\alt\\
                                  &   & \nonterm{forExpression}&\alt\\
                                  &   & \nonterm{caseExpression}&\alt\\
                                  &   & \nonterm{letExpression}&
    \end{array}
  \]
  \caption{Expression concrete syntax}
  \label{expressions}
\end{figure}

\FloatBarrier
\subsection{\texttt{skip} Expression}

Expression \lstinline|skip| can be used to designate a no-value when no action is needed (for example, in the body of unit which contains only declarations).

\subsection{Arrays, Lists, and S-expressions}

\begin{figure}[h]
  \[
    \begin{array}{rcl}
      \defterm{arrayExpression} & : & \term{[}\s[\s\nonterm{expression}\s(\s\term{,}\s\nonterm{expression}\s)^\star\s]\s\term{]}\\
      \defterm{listExpression}  & : & \term{\{}\s[\s\nonterm{expression}\s(\s\term{,}\s\nonterm{expression}\s)^\star\s]\s\term{\}}\\
      \defterm{S-expression}    & : & \token{UIDENT}\s[\s\term{(}\s\nonterm{expression}\s[\s(\s\term{,}\s\nonterm{expression}\s)^\star\s]\term{)}\s]
    \end{array}
  \]
  \caption{Array, list, and S-expressions concrete syntax}  
  \label{composite_expressions}
\end{figure}

There are three forms of expressions to specify composite values: arrays, lists and S-expressions (see Fig.~\ref{composite_expressions}).

\FloatBarrier

\subsection{Let Expressions}

\begin{figure}[h]
  \[
  \begin{array}{rcll}
    \defterm{letExpression} & : & \term{let}\s\nonterm{pattern}\s\term{=}\s\nonterm{expression}\s\term{in}\s\nonterm{expression}
  \end{array}
  \]
  \caption{Let-expression syntax}
  \label{let_expression}
\end{figure}

Let expression is a derived syntactic form for a one-branch case-expression. An expression

\begin{lstlisting}
   let p = e in b
\end{lstlisting}

is equivalent to

\begin{lstlisting}
   case e of
     p -> b
   esac
\end{lstlisting}

As let expression lacks an explicit ending specifier its scope extends to the right while possible; multiple let expressions on
the same nesting level associate to the right.


\FloatBarrier

\subsection{Conditional Expressions}

\begin{figure}[h]
  \[
    \begin{array}{rcll}
      \defterm{ifExpression}  & : & \term{if}\s\nonterm{expression}\s\term{then}\s\nonterm{scopeExpression}\s[\s\nonterm{elsePart}\s]\s\term{fi}&\\
      \defterm{elsePart}      & : & \term{elif}\s\nonterm{expression}\s\term{then}\s\nonterm{scopeExpression}\s[\s\nonterm{elsePart}\s]&\alt\\
                              &   & \term{else}\s\nonterm{scopeExpression}&
    \end{array}
  \]
  \caption{If-expression concrete syntax}
  \label{if_expression}
\end{figure}

Conditional expression branches the control depending in the value of a certain expression; the value zero is treated as falsity, nonzero as truth. An
extended form

\begin{lstlisting}
    if $\;c_1\;$ then $\;e1\;$
    elif $\;c_2\;$ then $\;e_2\;$
    ...
    else $\;e_{k+1}\;$
    fi
\end{lstlisting}

is equivalent to a nested form

\begin{lstlisting}
    if $\;c_1\;$ then $\;e1\;$
    else if $\;c_2\;$ then $\;e_2\;$
    ...
    else $\;e_{k+1}\;$
    fi
\end{lstlisting}

\FloatBarrier

\subsection{Loop Expressions}

\begin{figure}[t]
  \[
    \begin{array}{rcl}
      \defterm{whileDoExpression}  & : & \term{while}\s\nonterm{expression}\s\term{do}\s\nonterm{scopeExpression}\s\term{od}\\
      \defterm{doWhileExpression} & : & \term{do}\s\nonterm{scopeExpression}\s\term{while}\s\nonterm{expression}\s\term{od}\\
      \defterm{forExpression}    & : & \term{for}\s\nonterm{scopeExpression}\s\term{,}\s\nonterm{expression}\s\term{,}\s\nonterm{expression}\\
                                 &   & \term{do}\nonterm{scopeExpresssion}\s\term{od}
    \end{array}
  \]
  \caption{Loop expressions concrete syntax}
  \label{loop_expression}
\end{figure}

There are three forms of loop expressions~--- "\lstinline|while$\dots$do$\dots$od|", "\lstinline|do$\dots$while$\dots$od|", and "\lstinline|for$\dots$|", among
which "\lstinline|while$\dots$do$\dots$od|" is the basic one (see Fig.~\ref{loop_expression}). In "\lstinline|while$\dots$do$\dots$od|" expression the evaluation
of the body is repeated as long as the evaluation of condition provides a non-zero value. The condition is evaluated before the body on each iteration of the loop,
and the body is evaluated in the context of condition evaluation results.

The construct "\lstinline|do $\;e\;$ while $\;c\;$ od|" is derived and operationally equivalent to

\begin{lstlisting}
    $e\;$; while $\;c\;$ do $\;e\;$ od
\end{lstlisting}

However, the top-level local declarations in the body of "\lstinline|do$\dots$while$\dots$od|"-loop are visible in the condition expression:

\begin{lstlisting}
    do var x = read () while x od
\end{lstlisting}


The construct "\lstinline|for $\;i\;$, $\;c\;$, $\;s\;$ do $\;e\;$ od|" is also derived and operationally equivalent to

\begin{lstlisting}
    $i\;$; while $\;c\;$ do $\;e\;$; $\;s\;$ od
\end{lstlisting}

However, the top-level local definitions of the the first expression ("$i$") are visible in the rest of the construct:

\begin{lstlisting}
    for var i; i := 0, i < 10, i := i + 1 do write (i) od
\end{lstlisting}

\subsection{Pattern Matching}

\begin{figure}[t]
  \[
    \begin{array}{rcll}
      \defterm{pattern}         & : & \nonterm{consPattern}\alt\nonterm{simplePattern}&\\
      \defterm{consPattern}     & : & \nonterm{simplePattern}\s\term{:}\s\nonterm{pattern}&\\
      \defterm{simplePattern}   & : & \nonterm{wildcardPattern} & \alt\\
                                &   & \nonterm{S-exprPattern} & \alt \\
                                &   & \nonterm{arrayPattern} & \alt \\
                                &   & \nonterm{listPattern} & \alt \\
                                &   & \token{LIDENT}\s[\s\term{@}\s\nonterm{pattern} \s] & \alt \\
                                &   & [\s\term{-}\s]\s\token{DECIMAL}& \alt \\
                                &   & \token{STRING} & \alt \\
                                &   & \token{CHAR} & \alt \\
                                &   & \term{true} & \alt \\
                                &   & \term{false} & \alt \\
                                &   & \term{\#}\s\term{box} & \alt \\
                                &   & \term{\#}\s\term{val} & \alt \\
                                &   & \term{\#}\s\term{str} & \alt \\
                                &   & \term{\#}\s\term{array} & \alt \\
                                &   & \term{\#}\s\term{sexp} & \alt \\
                                &   & \term{\#}\s\term{fun} & \alt \\
                                &   & \term{(}\s\nonterm{pattern}\s\term{)} & \\
      \defterm{wildcardPattern} & : & \term{\_} &\\
      \defterm{S-exprPattern}   & : & \token{UIDENT}\s[\s\term{(}\s\nonterm{pattern}\s(\s\term{,}\s\nonterm{pattern})^\star\s\term{)}\s] &\\
      \defterm{arrayPattern}    & : & \term{[}\s[\s\nonterm{pattern}\s(\s\term{,}\s\nonterm{pattern})^\star\s]\s\term{]} &\\
      \defterm{listPattern}     & : & \term{\{}\s[\s\nonterm{pattern}\s(\s\term{,}\s\nonterm{pattern})^\star\s]\s\term{\}} &
    \end{array}
  \]
  \caption{Pattern concrete syntax}
  \label{pattern}
\end{figure}

Pattern matching is introduced into the language by the mean of \emph{case-expression} (see Fig.~\ref{case_expression}). A case-expression
evaluates an expression, called \emph{scrutinee}, and performs branching depending on its structure. This structure is specified by
means of \emph{patterns} (see Fig.~\ref{pattern}). If succeeded, a matching against a pattern delivers a
set of bindings~--- variables with their bindings to the (sub)values of the scrutinee.

The semantics of patterns is as follows:

\begin{itemize}
\item a pattern "\lstinline|$p_1$:$p_2$|" matches a list with a head matched with $p_1$ and a tail matched with $p_2$;
\item wildcard pattern "\lstinline|_|" matches every value;
\item S-expression pattern "\lstinline|$C\;$ ($p_1$,$\dots$,$\;p_k$)|" matches a value with corresponding top-level
  tag ("$C$") and arguments matched by subpatterns $p_i$ respectively; note, patterns can discriminate on the
  number of arguments for the same constructor, thus the same tag with different number of arguments can be
  used in different branches of the same case expression (see below);
\item array and list patterns match arrays and lists of the specified length with each element matched with
  corresponding subpattern;
\item an identifier matches every value and binds itself to that value in the corresponding branch of
  case-expression (see below);
\item a "\lstinline|$x$@$p$|"-pattern matches what pattern $p$ matches, and additionally binds the
  matched value to the identifier $x$;
\item constant patterns match corresponding constants;
\item six "\lstinline|#|"-patterns match values of corresponding shapes (reference values (\lstinline|box|), primitive values (\lstinline|val|),
  strings (\lstinline|str|), arrays, S-expressions or closures (\lstinline|fun|)) regardless their content;
  \item round brackets can be used for grouping.
\end{itemize}

All identifiers, occurred in a pattern, have to be pairwise distinct.

The matching against patterns in case-expression is performed deterministically in a top-down manner: a pattern
is matched against only if all previous matchings were unsuccessful. If no matching pattern is found, the execution
of the program stops with an error.

\begin{figure}[t]
  \[
    \begin{array}{rcl}
      \defterm{caseExpression}  & : & \term{case}\s\nonterm{expression}\s\term{of}\s\nonterm{caseBranches}\s\term{esac}\\
      \defterm{caseBranches}    & : & \nonterm{caseBranch}\s[\s(\s\term{$\mid$}\s\nonterm{caseBranch}\s)^\star\s]\\
      \defterm{caseBranch}      & : & \nonterm{pattern}\s\term{$\rightarrow$}\s\nonterm{scopeExpression}
    \end{array}
  \]
  \caption{Case-expression concrete syntax}
  \label{case_expression}
\end{figure}

%\subsection{Examples}
%\label{sec:expression_examples}
%
%Some other examples with comments:
%
%\begin{tabular}{ll}
%  "\lstinline|x !! y && z + 3|" & is equivalent to "\lstinline|x !! (y && (z + 3))|"\\
%  "\lstinline|x == y < 4|"      & invalid \\
%  "\lstinline|x [y := 8] := 6|" & is equivalent to "\lstinline|y := 8; x [8] := 6|"\\
%  "\lstinline|(write (3); x) := (write (4); z)|" & is equivalent to "\lstinline|write (3); write (4); x := z|"
%\end{tabular}






