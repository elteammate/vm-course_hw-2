% !TEX TS-program = pdflatex
% !TeX spellcheck = en_US
% !TEX root = lama-spec.tex
\FloatBarrier

\begin{figure}[t]
  \[
    \begin{array}{rcl}
      \defterm{compilationUnit}  & : & \nonterm{import}^\star\s\nonterm{scopeExpression}\\
      \defterm{import}           & : & \term{import}\s\token{UIDENT}\s\term{;}
    \end{array}
  \]
  \caption{Compilation unit concrete syntax}
  \label{compilation_unit}
\end{figure}

\section{Compilation Units}
\label{sec:compilation_units}

Compilation unit is a minimal structure recognized by the parser. An application can contain multiple units, compiled separately.
In order to use other units they have to be imported. In particular, the standard library is comprised of a number of precompiled units,
which can be imported by an end-user application.

The concrete syntax for compilation unit is shown in Fig.~\ref{compilation_unit}. Besides optional imports a unit must contain
a \nonterm{scopeExpression}, which may contain some definitions and computations. Note, a unit can not be empty. The computations described in
a unit are performed at the unit initialization time (see Chapter~\ref{sec:driver}).

